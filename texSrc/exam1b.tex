\documentclass[fleqn, a4paper, 12pt]{article}
\usepackage[utf8]{inputenc}
\usepackage[T1]{fontenc}
\usepackage{lmodern}
\usepackage[left=0.5in, right=0.5in, top=0.5in, bottom=0.5in]{geometry}
\usepackage{mathexam}
\usepackage{amsmath}
\usepackage{graphicx}

\let\ds\displaystyle

\newcommand{\me}{\mathrm{e}}



\ExamClass{Logical computational thinking}
\ExamName{I partial - exam B}
\ExamHead{01 October 2015}

\begin{document}
\ExamNameLine
\ExamStandardBoxEn
Harry is a potter, he works on commission and he receive orders of
a certain quantity of pottery to make in a certain amount of
days.

Harry knows that he is capable of doing 10 objects in a day and
no more. He want a program to helps him to know if is able to
fulfill a certain order or not. So before accepting an order he inserts
the quantity and the days in the program, and he wants to know if he
can produce that quantity in those days or not.

Remember that if $q$ is the quantity and $d$ is the days, you need to
keep a rhythm of at least:
\begin{center}
  \begin{equation*}
    r = \frac{q}{d}
  \end{equation*}
\end{center}
ceramics per day.
\newpage
\ExamNameLine
\ExamStandardBoxEs
Harry es un alfarero, el trabaja por encargos y recibe ordenes de una
cierta quantidad de alfarer\'ia da hacer en un cierto numero de dias.

Harry sabe que es capaz de hacer 10 objetos al dia y nada mas. El
quiere un programa que lo ayude a saber si es capaz de cumplir una
horden o menos. Por esto antes de acceptar una horden el inserta la
quantidad y los dias en el programa, y quiere saber si el puede
fabricar esa quantidad en esos dias o no.

Requerda que si $q$ es la quantidad y $d$ los dias, necesitas mantener
un ritmo de por lo menos:
\begin{center}
  \begin{equation*}
    r = \frac{q}{d}
  \end{equation*}
\end{center}
cer\'amicas por dia.
\end{document}

