%This work is licensed under the Creative Commons
%Attribution-ShareAlike 4.0 International License. To view a copy of
%this license, visit http://creativecommons.org/licenses/by-sa/4.0/ or
%send a letter to Creative Commons, PO Box 1866, Mountain View, CA
%94042, USA.

\input{header}

\title[Lesson 4]{\textbf{Lesson 4 - Shell and \texttt{gcc} basics}}
\date[17/9/15]{\flushright 17 September 2015}

\begin{document}

\begin{frame}[plain]
  \titlepage
\end{frame}

\begin{frame}
  \frametitle{\bash\ shell}
  \begin{block}{What is a shell}
    Is a program that interprete commands given by the user. \C\
    programs can use the shell for basic input/output.
  \end{block}
  \begin{block}{important commands}
    \begin{itemize}
    \item \bb{pwd}: stands for Print Working Directory, print the
      current path, with directories separated by \bb{/}
    \item \bb{ls}: list the content of the current directory, you can
      add the option \bb{-l} for detailed output and \bb{-a} for
      seeing also hidden files
    \item \bb{cd path}: change the current directory to \bb{path}, can
      be also a \bb{nested/path} and \bb{cd /} go to the root, \bb{cd
        ~} or \bb{cd} go to the user's home (in Cygwin the home is not
      the windows home)
    \item \bb{./name}: for launching an executable called \bb{name}
      inside current path
    \end{itemize}
  \end{block}
\end{frame}

\begin{frame}
  \frametitle{\gcc\ (GNU C compiler)}
    \begin{block}{What is a compiler}
      Transform a source code in something executable from the
      machine.
    \end{block}
    \begin{center}
      \begin{tikzpicture}[node distance=17mm]
        \node (source)
        [file, label=above:\texttt{sName.c}, fill=c1]
        {\C\ source code};
        \node (obj)
        [file, label=above:\texttt{oName.o}, right=of source, fill=c2]
        {Object file};
        \node (libs)
        [files, label={[label distance=1mm]above:\texttt{libs.h}}, below=7mm of obj, fill=c3]
        {Libraries};
        \node (exe)
        [file, label=above:\texttt{eName\{.exe\}}, right=of obj, fill=c4]
        {Executable};
        \draw [arrow] (source) -- node [above, color=red] {compile} (obj);
        \draw [arrow] (obj) -- node [above, color=red] {link} (exe);
        \draw [arrow] (libs.east) -- ($ (exe.west) - (8mm,0) $);

      \end{tikzpicture}    
    \end{center}
    \begin{itemize}
    \item \alert{Compile:} \bb{gcc -c sName.c -o oName.o}; if \bb{-o}
      option not present, automatically use \bb{sName.o} as name
    \item \alert{Link:} \bb{gcc oName.o -o eName}; if \bb{-o}
      option not present, automatically use \bb{a.out} as name
    \item \underline{\alert{Compile + link:} \bb{gcc sName.c -o eName}}
    \end{itemize}
\end{frame}

\begin{frame}
  \frametitle{Steps for building a program}
  \begin{enumerate}
  \item Open a \alert{text editor}
  \item\label{point:write} \alert{Write} the code (or open and modify an existing one)
  \item\label{point:path} \alert{Save} the file in a known path and with the
    extension \bb{.c} (i.e. \bb{name.c})
  \item Open the \alert{shell} (Cygwin for Windows, Terminal for Mac)
  \item Go to the same \alert{path} of the point~\ref{point:path}; use
    \bb{cd} and remember that the path to \texttt{Documents} is:
    \begin{itemize}
    \item for Windows with Cygwin: \texttt{/cygdrive/c/Users/[YourName]/Documents}
    \item for Mac: \texttt{\~/Documents}
    \end{itemize}
  \item \alert{Compile} the program with \bb{gcc name.c -o name}
  \item \alert{Execute} the program with \bb{./name}
  \item If you are happy with the result finish, otherwise go to
    point~\ref{point:write} with the same file
  \end{enumerate}
\end{frame}
\end{document}
