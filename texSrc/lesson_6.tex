%This work is licensed under the Creative Commons
%Attribution-ShareAlike 4.0 International License. To view a copy of
%this license, visit http://creativecommons.org/licenses/by-sa/4.0/ or
%send a letter to Creative Commons, PO Box 1866, Mountain View, CA
%94042, USA.

\input{header}

\title[Lesson 6]{\textbf{Lesson 6 - Boolean logic and iterations}}
\date[15/10/15]{\flushright 15 October 2015}

\begin{document}

\begin{frame}[plain]
  \titlepage
\end{frame}

\begin{frame}
  \frametitle{Boolean logic}
  \begin{block}{Predicate}
    A function
    $$
    P:X\rightarrow \{true,false\}
    $$
    from a certain set $X$ (for istance $\mr\times\mr$) to a truth value. Can be: \cc{<}, \cc{>},
    \cc{<=} ($\leq$), \cc{>=} ($\geq$), \cc{==} ($=$), \cc{!=} ($\neq$)
    
  \end{block}
  \begin{block}{Logical connective}
    Technically predicates in the set:
    $\{true,false\}\times\{true,false\}$ ($\{true,false\}$ for the
    negation), can connect different expressions together. can be:
    \cc{\&\&} ($\wedge$), \cc{||} ($\vee$), \cc{!} ($\neg$).
    \begin{center}
      \begin{tabular}{|c|c||c|c|c|}
        \hline
        $a$ & $b$ & $\neg a$ & $a\wedge b$ & $a\vee b$ \\
        \hline
        0 & 0 & 1 & 0 & 0 \\
        0 & 1 & 1 & 0 & 1 \\
        1 & 0 & 0 & 0 & 1 \\
        1 & 1 & 0 & 1 & 1 \\
        \hline
      \end{tabular}
    \end{center}
  \end{block}
\end{frame}
\begin{frame}
  \begin{block}{Boolean expression}
    An expression that produce a boolean value when evaluated (true,
    false). Can be composed from
    \begin{itemize}
    \item variables
    \item predicates
    \item connectives
    \item parenthesis
    \end{itemize}
  \end{block}
  for instance:
  \begin{itemize}
  \item \cc{a<3}
  \item \cc{(a>5) \&\& (a<10)}
  \item \cc{(a<b) || !(a>=10 \&\& b<=5)}
  \end{itemize}
\end{frame}

\begin{frame}
  \frametitle{Iterations}
  \begin{center}
    \begin{tikzpicture}[node distance=5mm, font=\tiny, auto]
      \node(start1) [fStartEnd] {Start};
      \node(selection1) [fSelection, below=of start1] {\cc{Cond}};
      \draw [arrow] (start1) -- (selection1);
      \node(iterBlock1) [fProcess, below=of selection1] {\cc{iterBlock}};
      \draw [arrow] (selection1) -- node [near start] {true} (iterBlock1);
      \node(end1) [fStartEnd, below=15mm of iterBlock1] {End};
      \draw [arrow] (iterBlock1)
      -- ($ (iterBlock1.south) - (0,5mm) $)
      -| ($ (selection1.west) - (5mm,0) $)
      |- ($ (selection1.north) + (0,2.5mm) $);
      \draw [arrow] (selection1)
      -- node [near start] {false} ($ (selection1.east) + (5mm,0) $)
      |- ($ (end1.north) + (0,5mm) $)
      -- (end1);
      
      \node(start2) [fStartEnd, right=50mm of start1] {Start};
      \node(iterBlock2) [fProcess, below=of start2] {\cc{iterBlock}};
      \draw [arrow] (start2) -- (iterBlock2);
      \node(selection2) [fSelection, below=of iterBlock2] {\cc{Cond}};
      \draw [arrow] (iterBlock2) -- (selection2);
      \draw [arrow] (selection2) 
      -- node [near start] {true} ($ (selection2.west) - (5mm,0) $)
      |- ($ (iterBlock2.north) + (0,2.5mm) $);
      \node(end2) [fStartEnd, below=of selection2] {End};
      \draw [arrow] (selection2) -- node [near start] {false} (end2);
    \end{tikzpicture}
  \end{center}
\end{frame}

\end{document}
