\documentclass[fleqn, a4paper, 12pt]{article}
\usepackage[utf8]{inputenc}
\usepackage[T1]{fontenc}
\usepackage{lmodern}
\usepackage[left=0.5in, right=0.5in, top=0.5in, bottom=0.5in]{geometry}
\usepackage{mathexam}
\usepackage{amsmath}
\usepackage{graphicx}

\let\ds\displaystyle

\newcommand{\me}{\mathrm{e}}



\ExamClass{Logical computational thinking}
\ExamName{I partial - exam A}
\ExamHead{01 October 2015}

\begin{document}
\ExamNameLine
\ExamStandardBoxEn
A couple of gardeners called Alice and Bob asked you to develop a program for helping Them
comparing the areas of different
flowerbeds with rectangular and circular shapes and varying dimensions. Alice work in pair
with Bob, and they always work simultaneously in rounds, one trimming in a rectangular
flowerbed and one in a circular flowerbed.

The gardeners want to be
fair and decided to organizing the work in a manner that if in one
round Alice worked in a bigger flowerbed than the one of Bob, in the next round
will be Bob to work in a bigger flowerbed respect Alice. For doing
that they need your program, they want to insert the length of the
edges of the rectangular flowerbed, the length of the diameter of
the circular flowerbed, and they want to know which one of the two
flowerbeds is bigger.

Remember that the area of a rectangle with edges $a$ and $b$ is:
\begin{center}
  \begin{equation*}
    A_r = a\cdot b
  \end{equation*}
\end{center}
and the area of a circle with diameter $d$ is:
\begin{center}
  \begin{equation*}
    A_c = \frac{\pi \cdot d^2}{4}.
  \end{equation*}
\end{center}
\newpage
\ExamNameLine
\ExamStandardBoxEs
Una pareja de jardineros llamados Alice y Bob the pedieron de
desarrollar un programa por ayudarlos a comparar las areas de
diferentes parterre rectangulares y circulares, con medidas
variables. Alice trabaja en pareja con Bob, y siempre trabajan
simultaneamente en rounds, uno recortando en un parterre rectangular y
uno en un parterre circular.

Los jardineros quieren ser justos y decidieron organizar el trabajo de
manera que si en un round Alice ha trabajado en un parterre mas grande
de lo de Bob, en el siguiente round ser\`a Bob a trabajar en un
parterre mas grande respecto a Alics. Por acer esto necesitan tu
programa, ellos quieren insertar lo largo de los bordes de el
parterre rectangular, lo largo de el diametro de el parterre circular,
y quieren saber qual es el parterre mas grande.

Requerda que la area de un rectangulo con latos $a$ y $b$ es:
\begin{center}
  \begin{equation*}
    A_r = a\cdot b
  \end{equation*}
\end{center}
y la area de un circulo de diametro $d$ es:
\begin{center}
  \begin{equation*}
    A_c = \frac{\pi \cdot d^2}{4}.
  \end{equation*}
\end{center}
\end{document}
