%This work is licensed under the Creative Commons
%Attribution-ShareAlike 4.0 International License. To view a copy of
%this license, visit http://creativecommons.org/licenses/by-sa/4.0/ or
%send a letter to Creative Commons, PO Box 1866, Mountain View, CA
%94042, USA.

\input{header}

\title[Lesson 3]{\textbf{Lesson 3 - Compiling a \C\ program}}
\date[10/9/15]{\flushright 10 September 2015}

\begin{document}

\begin{frame}[plain]
  \titlepage
\end{frame}

\begin{frame}
  \frametitle{\texttt{gcc} compiler, GNU/Linux}
  Installed and working by default on every default.
\end{frame}

\begin{frame}
  \frametitle{\texttt{gcc} compiler, Mac}
  \footnotesize
  \begin{block}{Install}
    \begin{enumerate}
    \item Go to \url{https://developer.apple.com/downloads/};
    \item provide authentication, and if needed accept the agreement;
    \item search \alert{command line tools} on the left box;
    \item double click on \alert{Command line tools (...) for Xcode
        6.4};
    \item download \alert{dmg} file (click on the filename on the right);
    \item double click on the \alert{dmg} and then double
      click on the \alert{pkg} inside it;
    \item follow the installation instructions.
    \end{enumerate}
  \end{block}
  \begin{block}{First use}
    \begin{enumerate}
    \item Open a terminal;
    \item type \alert{\texttt{sudo gcc -v}};
    \item provide authentication;
    \item read license, with \alert{space}, and agree, typing
      \alert{\texttt{agree}} at the end (attention to not press to
      much spaces).
    \end{enumerate}
  \end{block}
\end{frame}

\begin{frame}
  \frametitle{\texttt{gcc} compiler, Windows}
  \begin{block}{Install}
    \begin{enumerate}
    \item Go to
    \end{enumerate}
  \end{block}
\end{frame}
\end{document}
