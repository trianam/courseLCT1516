%This work is licensed under the Creative Commons
%Attribution-ShareAlike 4.0 International License. To view a copy of
%this license, visit http://creativecommons.org/licenses/by-sa/4.0/ or
%send a letter to Creative Commons, PO Box 1866, Mountain View, CA
%94042, USA.

\include{header}

\title[Lesson 2]{\textbf{Lesson 2 - Introduction to C language}}
\date[10/9/15]{\flushright 10 September 2015}

\begin{document}

\begin{frame}[plain]
  \titlepage
\end{frame}

\begin{frame}
  \frametitle{Review}
  \begin{center}
    \begin{tikzpicture}[node distance=5mm]
      \node (realProb) [oval, fill=c1] {Problem/Objective};
      \node (model) [rect, fill=c2, below=of realProb] {Modeling};
      \node [comment, right=of model] {Input/output, limits, \dots};
      \draw [arrow] (realProb) -- (model);
      \node (algorithm) [rect, fill=c2, below=of model] {Algorithm};
      \node [comment, right=of algorithm] {Flow charts, pseudocode};
      \draw [arrow] (model) -- (algorithm);
      \node (implementation) [rect, fill=c2, below=of algorithm] {Implementation};
      \node [comment, right=of implementation] {Design, Coding};
      \draw [arrow] (algorithm) -- (implementation);
      \node (application) [oval, fill=c1, below=of implementation] {Application};
      \draw [arrow] (implementation) -- (application);
      \draw [arrow] ($ (implementation.south) + (0,-2mm) $) -- ++(-20mm,0) |- ($ (model.north) + (0,4mm) $);
      \draw [arrow] ($ (algorithm.south) + (0,-2mm) $) -- ++(-15mm,0) |- ($ (model.north) + (0,2mm) $);
    \end{tikzpicture}
  \end{center}
\end{frame}

\begin{frame}
  \frametitle{Model}
  \begin{center}
    \begin{tikzpicture}[node distance=20mm]
      \node (inputs) [comment, rotate=45] {Inputs};
      \node (algorithm) [clo, fill=c1, right=of inputs] {Algorithm};
      \node (outputs) [comment, right=of algorithm, rotate=315] {Outputs};
      \draw [arrow] (inputs.north east) -- (algorithm);
      \draw [darrow] (inputs.east) -- (algorithm);
      \draw [arrow] (inputs.south east) -- (algorithm);
      \draw [arrow] (algorithm) -- (outputs.north west);
      \draw [darrow] (algorithm) -- (outputs.west);
      \draw [arrow] (algorithm) -- (outputs.south west);
    \end{tikzpicture}    
  \end{center}
\end{frame}

\begin{frame}
  \frametitle{Algorithm}
  \begin{center}
        \begin{tikzpicture}[node distance=5mm, font=\tiny, auto]
      \node(start) [fStartEnd] {Start};
      \node(input) [fInput, below=of start] {$A,B,C$};
      \draw [arrow] (start) -- (input);
      \node(opD) [fProcess, below=of input] {$\Delta = B^2-4\cdot
        A\cdot C$};
      \draw [arrow] (input) -- (opD);
      \node(selection) [fSelection, below=of opD] {$\Delta\ge 0$?};
      \draw [arrow] (opD) -- (selection);
      \node(opX1T) [fProcess, below=of selection]
      {$x_1=(-B+\sqrt{\Delta})/(2\cdot A)$};
      \draw [arrow] (selection) -- node [near start] {yes} (opX1T);
      \node(opX2T) [fProcess, below=of opX1T]
      {$x_2=(-B-\sqrt{\Delta})/(2\cdot A)$};
      \draw [arrow] (opX1T) -- (opX2T);
      \node(outputErr) [fOutput, right=10mm of opX1T] {ERROR\\(no real
      solutions)};
      \draw [arrow] (selection) -| node [near start] {no} (outputErr);
      \node(output) [fOutput, below=of opX2T] {$x_1, x_2$};
      \draw [arrow] (opX2T) -- (output);
      \node(end) [fStartEnd, below=of output] {End};
      \draw [arrow] (output) -- (end);
      \draw [arrow] (outputErr) |- ($ (end.north) + (0,2mm) $);
    \end{tikzpicture}
  \end{center}
\end{frame}

\begin{frame}
  \frametitle{C language}
  \begin{figure}[h!]
    \centering
    \includegraphics[width=5cm]{img/ken_n_dennis.jpg}
    \caption{Ken Thompson and Dennis Ritchie}
  \end{figure}
  \begin{block}{History}
    \begin{itemize}
    \item The language \alert{C} is developed in the 70's by Dennis Ritchie
    \item Along with the \alert{Unix} system, created by Ken Thompson and
      Dennis Ritchie in the same years
    \end{itemize}
  \end{block}
\end{frame}

\begin{frame}[fragile]
  \frametitle{Basis}
    \begin{cblock}
#include<stdio.h>  //library 

void main() {  //begin of main
  printf("Hello world!\n");
}
\end{cblock}
\cfile{../cSrc/hello_world.c}
\end{frame}
\end{document}